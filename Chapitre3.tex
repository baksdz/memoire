\chapter{État de l'art des techniques}
\chaptermark{Techniques}
\minitoc
\newpage
 
\section{Introduction}

Il existe Plusieurs indices pour mesurer la qualité d'une classification
non supervisée. Ils sont principalement de deux sortes; les indices
externes et les indices internes. L'utilisation des indices externes,
requière le recours à des information à priori sur les classes. Tandis
que les critères internes, reposent sur la cohérence de la répartition
des données dans la classification (similarité, dé-similarité, inertie
des classes et cohésion topographique).

Dans cette partie, nous allons passer en revue les principaux indices de
qualité. Nous verrons en premier lieu, les indices externes et dans un
second les internes. Pour les premiers, nous allons nous intéresser au
taux global d'erreur de classification (Overall Error Rate of
Classification OERC) sans coût et avec coût, sections 3.1.1 et 3.1.2. En
section 3.1.3 nous aborderons les courbes ROC (Receiver Operating
Characteristics) et à l'Aire sous courbes ROC (Area Under the Curve AUC)
en section 3.1.4. Et enfin nous traiterons les mesures de Rappel,
Précision et F-mesure (Precision, Recall and F-measure), respectivement
en section 3.1.5, 3.1.6 et 3.1.7. En second lieu, nous nous
intéresserons au mesure internes. indice de Dunn (1974) section 3.2.1
indice BH de Baker et Hubert (1975) section 3.2.2 indice DB de Davies et
Bouldin (1979) section 3.2.3 indice CH de Calinsky et Harabsz (1974)
section 3.2.4 indice Silhouette (S) de Rousseeuw (1987) section 3.2.5
indice HL de Hubert et Levin (1976) section 3.2.6 indice KL de
Krzanowski et Lai (1988) section 3.2.7

\section{Indices externes}\label{indices-externes}

\subsection{Taux global d'erreur de classification}\label{taux-global-derreur-de-classification}
Salah dhjklfdjsdl ~\cite{hajjar}.

%Critères le plus trivial de la mesure de la qualité de classification.
Il consiste a rapporter l'ensemble des observations mal classées par
rapport à la totalité des observations classées. Il peut être calculé à
partir du taux global de bonnes classifications (\(T_{gbc}\)).

\(T_{gec} = 1 - T_{gbc}\)

avec

\(T_{gbc} = \frac{1} {n}\sum_{i=1}^k n_{i,i}\)

ou

\(T_{gec} = \frac{1} {n}\sum_{i,j=1}^k n_{i,j}\)\texttt{~avec~}\(i \neq j\)

Cet indice reste pertinent, tant que les classes sont équilibrées en
termes de taille et de répartition. Cependant, il est à éviter quand
nous avons à faire à des classes de tailles très variées. En effet, si
notre set de données est constitué deux deux classes, l'une de taille
petite et l'autre bien plus grande, un modèle qui classerait l'ensemble
dans la deuxième, aura un taux d'erreur faussement bas. Pour remedier à
ce biais, La modération du poids des grosses classe

\begin{itemize}
\tightlist
\item
  taux d'erreur dans la classification (peu pertinent dans certains
  cas). cet indice est peux efficace quand il s'agit de qualifier une
  classification avec des classe déséquilibrées...et peut donner une
  évaluation éronné de la qualité du classifieur dans le cas ou on a un
  jeu de données avec des classes très petite et des classes très
  grandes.
\end{itemize}

\emph{Cette anomalie est liée au fait que nous voulons absolument que le
modèle réalise un affectation (positif ou négatif). Or dans de nombreux
domaines, ce qui nous intéresse avant tout, c'est de mesurer la
propension à être positif ou négatif !}(propension=tendance à)

\subsection{le taux d'erreur avec
coût}\label{le-taux-derreur-avec-couxfbt}

Afin de palierà cette faille, il a été introduit aurailes classe sont de
tailles très différentes (ce qui est souvent le cas). En effet, cet
indice donnerait un taux d'erreur très basfaible quand il s'agit de
trouver des petites classes rappel précision f-mesure intéressant car
ils permettent un ajustement du modèle (plus de rappel pour ou alors
plus de précision pour)

Rappel result+/total+ {[}TP/(TP+FN){]} plus le rappel est proche de 1
plus le nombre de Positif trouvé Précision result+/totalc
{[}TP/(TP+FP){]} si on veut éviter le FP F-mesure moyenne harmonique de
Rappel et précision

\subsection{Matrice de confusion}\label{matrice-de-confusion}

\subsection{Courbes ROC}\label{courbes-roc}

\subsection{Aire sous courbes ROC
AUC}\label{aire-sous-courbes-roc-auc}

\subsection{précision rappel}\label{pruxe9cision-rappel}

=

\subsection{F-mesure}\label{f-mesure}

\section{Indices Internes}\label{indices-internes}

\subsection{indice de Dunn (1974)}\label{indice-de-dunn-1974}

\subsection{indice BH de Baker et Hubert
(1975)}\label{indice-bh-de-baker-et-hubert-1975}

\subsection{indice DB de Davies et Bouldin
(1979)}\label{indice-db-de-davies-et-bouldin-1979}

\subsection{indice CH de Calinsky et Harabsz
(1974)}\label{indice-ch-de-calinsky-et-harabsz-1974}

\subsection{indice Silhouette (S) de Rousseeuw
(1987)}\label{indice-silhouette-s-de-rousseeuw-1987}

\subsection{indice HL de Hubert et Levin
(1976)}\label{indice-hl-de-hubert-et-levin-1976}

\subsection{indice KL de Krzanowski et Lai
(1988)}\label{indice-kl-de-krzanowski-et-lai-1988}


\section{Conclusion}